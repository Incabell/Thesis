\documentclass{article}
\usepackage{prerex}
\usepackage{parselines}
\usepackage{array}

\smash{\Huge}

\date{11.11.2015}

\begin{document}

\section{Sceptic long-time editor - Primary Persona}

\subsection{Demography:}
\begin{itemize}
 \item male
 \item 28 years old
 \item lives in Mannheim, Germany
 \item single for 2 years
 \item studied economics at Uni Mannheim
 \item lives in a single apartment in quite neighborhood
 \item works at a white collar job in a non-leading position
\end{itemize}

\subsection{Behaviors:}
\begin{itemize}
\item gets into heated discussions a lot
\item has problems handing of control
\item thinks his opinion is being overheard a lot of the time
\item very suspiscious of people and their Behaviors
\item avid twitter and facebook user
\item likes to comment on most things to display his opinion
\end{itemize}

\subsection{Needs and Goals:}
\begin{itemize}
 \item is frustrated because he feels like he doesn't get all the respect he deserves
 \item wants more control and responsibility
 \item wants to be admired
 \item wants to have smth to show for
 \item worries that he won't "make it" in life
 \item scared of being compared to other people
\end{itemize}

\pagebreak

https://www.wikidata.org/wiki/User:Sicherlich
\begin{quote}
"If you find (how ever) your way from the Wikipedia-Artikel Warszawa to Q270 - then what? If I would like to use the coordinates then what am I supposed to do? clicking on it? - no. edit? - no. Well so what? how? ? - the free encyclopedia that anyone can edit...."
\end{quote}

Sven is a 28 year old male who works as an accountant in a big company. He is the younger brother of two and always strives to be as successfull as his older brother. He doesn't like the level of responsibility he has at work and always tries to impress everyone with his knowledge. Usually his co-workers ignore what he has to say or don't take it too seriously. That really upsets Sven but he feels like he's powerless against it. He feels like he's really good at his job and deserves a promotion. He wishes that his co-workers would aknowledge his abilities and thus would look up to him more. When his brother was his age he was already head of a department of his company. That really bothers him. 
When he gets home from work he tries to meet a friend for drinks but he's busy. He feels like lately his friends have less and less time for him because they all seem to be so occupied with their jobs and their relationships. Since his relationship ended two years ago, he has had a hart time meeting the right person. This only adds up to his frustration. He knows that it isn't helathy but he doesn't know what to do about it. Even in his editing community he isn't very respected and he's been editing for over 6 years. He likes to participate in discussion because he feels like that way he can make a change, but often the reaction to his comments isn't very positive. 
He hates it when people mess with his articles. He doesn't like sharing the editing powers although he knows that Wikipedia is a community contribution but he feels like it's almost a personal violation when his words are being edited. He usually maintains the english and german Wikipedia articles and is often annoyed that he has to retype the change twice when things change. He is very sceptical about the infobox information coming from Wikidata but on the other hand it would really relieve him from a lot of typing. 

\section{User scenario:}
\subsection{Expectations and desires:}
\begin{itemize}
\item wants to keep a say in what will be written there
\item doesn't want it to be easier for other people to edit his article
\item super easy, straight forward Wikidata integration
\item doesn't want to be made feel stupid
\item no transfer to Wikidata itself!
\end{itemize}

\subsection{Context Scenario:}
\begin{enumerate}
\item When Sven comes home from work he goes to check on his ongoing discussions with other editors. He sees that someone edited something in one of his articles again. There has been an ongoing discussion for a while for a specific value in the infobox on which no one can agree. Sven decides that it's time to settle this now and choses to finally implement Wikidata for his infoboxes. He opens the visualEditor and uses the convert to get information from Wikidata button. All the properties that are available are changed to receive their information from there. Fortunately for Sven the value which was disputed, is the same as his in the Wikidata item, so it is now changed back to his desired value.  
\item Later that evening Sven realises that an article he follows doesn't have an infobox so he decides to add one. He opens the VisualEditor and decides to just let it automatically generate it, although he doesn't fully trust that yet but it's just one button click away and he can still edit all the information afterwards if neccesary. 
\item He is surprised at how easily the changes are possible without requiring any knowledge about Wikidata. He decides to go through some more infoboxes to see if Wikidata is being used yet. He finds one where most values are still manually written in. He sets all possible boxes to 'get from Wikidata'. Before hitting the 'save' button he can see how it would look with the values to make sure everything is correct. He really likes that feature so he doesn't have to commit to something that might be wrong. 
\end{enumerate}

\end{document}
