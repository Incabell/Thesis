\documentclass{article}
\usepackage{prerex}
\usepackage{parselines}
\usepackage{array}

\smash{\Huge}

\date{11.11.2015}

\begin{document}

\section{Sceptic long-time editor - Primary Persona}

\subsection{Demography:}
\begin{itemize}
 \item female
 \item 36 years old
 \item lives in Heidelberg, Germany
 \item married
 \item studied veterinary medicine Uni Heidelberg
 \item lives in a single apartment in quite neighborhood
 \item works as a veterianrian at Uniklinik Heidelberg
 \item has two dogs
 \item apartment in city center
\end{itemize}

\subsection{Behaviors:}
\begin{itemize}
\item calm personality
\item very assertive
\item very sociable
\item pragmatic/ goal oriented
\item helper personality
\item spends a lot of time in forums answering users questions
\end{itemize}

\subsection{Needs and Goals:}
\begin{itemize}
 \item protecting animals
 \item spread awareness
 \item likes the feeling to be needed and help
 \item would like more regular working hours
 \item worries that her marriage might fall apart
 \item scared of failing at her job and thus hurting someone
\end{itemize}

\pagebreak

https://www.wikidata.org/wiki/User:fillWithUsername
\begin{quote}
"Fill with quote"
\end{quote}

Miriam is 36 year old veterinarian working at the University clinic of Heidelberg. She often has to work long shifts and sometimes also night shifts. Often when she comes home her husband is already asleep and so their marriage hasn't been very stable for a while. She's thinking of changing to a regular veterinarian's office so she can have normal working hours. She lives for her job because helping animals is what drives her. When she sees a suffering animal, that needs help, she is unable to resist. She tries to extend her knowledge of animals, their illnesses and their potential treatments to the outside world because she wants that as many animals as possible find help. That's the reason why she spends hours in forums, when she comes home from work, answering people's questions about their animals. A couple of years ago she discovered Wikipedia as a tool to also spread knowledge. When she started editing she found that many articles on animal illnesses were incomplete or entirely missing. Over the years she has written and edited 20+ articles hoping that someone will make use out of them. Most of her articles are rather scientific and thus often require an infobox. Miriam knows about Wikidata but has no idea how to integrate the information and no time to figure it out. Right now she always types everything by hand but she would appreciate the help of not having to do so by just integrating Wikidata's item. She also already encountered the situation where there was an english version of her article but she didn't know how to connect both of them. She often worries that people don't take care properly of their pets. She has seen many situations in her work where mistreat them without even them knowing it. 
She sometimes thinks about having kids, but the way her marriage is going she doesn't think that would be a good idea. She feels bad for not being home often and letting her marriage slip but she is incapable of giving up on her animal care. She is determined to make a difference, even when it is just one article at a time.
She loves her two dogs Lilo and Stitch. She tries to spend as much time with them as possible taking them for extensive walks as often as she can and taking them on her run every morning. 

\section{User scenario:}
\subsection{Expectations and desires:}
\begin{itemize}
\item make editing easier
\item less information to update
\item save her time
\item easily understandable
\item sitelinking shoulb be easier
\item would appreciate an auto-generation button for the infobox
\end{itemize}

\subsection{Context Scenario:}
\begin{enumerate}
\item After Miriam comes home from work she sits down to answer questions that people might have asked her in one of the many forums that she's active in. In response to a specific issue she wants to link the relevant Wikipedia article. She notices a small mistake in the infobox though and quickly wants to change it, before sending the link. She logs in and opens the VisualEditor as usual. She sees that the value used in there is still typed in by hand. She much prefers the infobox to use Wikidata so she goes ahead and switches the box to "use Wikidata item for this value" and when she saves the correct value for that property it's displayed in the infobox.
\item She sits down to do some research on a disease she encountered today at the vet's office. She finds some information in veterinary journals but sadly no Wikipedia article. She knows that people would never read those journals to get informed about a disease so she decides to write an article. She writes the text first which she derives from multiple of the scientific papers she has read on this issue. It always takes so much time to write the articles acurately and not get any of the facts wrong so all she really wants is to have to do as little as possible. The new function of auto-generating an infobox is exactly what she wants. She just has to oben the visual editor and click on create infobox from Wikidata item, type in the Item ID and the infobox template she wants to use and everythign is generated for her. She then goes through all the information to make sure it's correct before saving. 
\item There is a short article on the english Wikipedia, so she wants to make sure later, to link those two together. She used to have to do this manually, but because she used the Item in the infobox generation it automatically suggested to her to add this article to the item since it realised that there is no german article linked to this item yet. She clicks yes and is relieved that yet another task is taken of her hands. She gets to go to bed earlier than expected which makes her very happy.
\end{enumerate}

\end{document}
