\documentclass{article}
\usepackage{prerex}
\usepackage{parselines}
\usepackage{array}

\smash{\Huge}

\date{11.11.2015}

\begin{document}

\section{Positive long-time editor - Primary Persona}

\subsection{Demography:}
\begin{itemize}
 \item male
 \item 44 years old
 \item lives in seattle, washington
 \item married to Tom Cohen
 \item two daughters: Franscene and Rebecca
 \item science teacher at local highschool
 \item speaks english and french
 \item likes hiking and birdwatching
 \item organzies editing sessions in his community
 \item knows Wikidata but has a hard time using it
\end{itemize}

\subsection{Behaviors:}
\begin{itemize}
\item involved in multiple wiki projects
\item admin/editor on a larger scale
\item very community oriented
\item likes to help people
\item likes to organize things
\item likes time to himself
\item fatherly figure
\end{itemize}

\subsection{Needs and Goals:}
\begin{itemize}
 \item want to see the projects grow and improve 
 \item expanding free knowledge
 \item help people have acces to free knowledge
 \item concerned wether he's making a difference
 \item frustrated about the senseless vandalism 
 \item fears that Wikipedia will stagnate at some point
\end{itemize}

\pagebreak

https://www.wikidata.org/wiki/User:Montanabw
\begin{quote}
"In en-wiki, there is no way any ordinary editor would have the slightest clue when creating an article that wikidata even exists and even less than the slightest clue how to create it. I've been editing nine years, nothing to help me other than talk page chitchat and completely incomprehensible instructions. Wiki markup syntax is complex enough for non-programmers, to go to a different web site and input something is too steep a learning curve for me..."
\end{quote}


Jeff is 44 years old and works as a science teacher at a highscool mainly teaching biology and chemistry. He has been teaching for almost 20 years and he still enjoys his job. He sometimes worries that the kids nowadays take the access to free knowledge for granted and thus wants to make sure that there will be enough future contributors out there. He runs a monthly open editing session to which he encourages his students to come as well. He has been involved in community work for a long time and is a respected member. He is not the most tech-savy person but that hasn't stopped him from learning the Wiki markup syntax. Fortunately since December 2012 he can use the visual editor which makes it much easier for him to maintain his pages. He loves the idea of Wikidata and would love to see the community use it more avidly but he struggles with the usage and often can't be bothered to find out how to work it from the little instructions he can gather scattered around different pages which results in him refering to the old ways.
On the weekends he likes to go hiking and do some birdwatching with his husband Tom, and his daughters, Fran and Beccy. When they get home he usually cooks dinner for everyone. He likes to care for his family and make sure that everyone is doing well. He always encourages them to ask questions and makes sure they understand where knowledge comes from. He wants them to uderstand that it is something that has to be maintained and protected. 

\section{User scenario:}
\subsection{Expectations and desires:}
\begin{itemize}
\item keep vandalism low
\item but also low entry barrier
\item should be usable with the knowledge he already has
\item doesn't want too much to change since it took him so long to learn everything
\item should be instantly doable
\item should be the same language
\item showcase the connection with WD
\item explain what WD is briefly
\item easily understandable tutorials and explanations should be offered
\item should clearly differentiate that this is WD data
\end{itemize}

\subsection{Context Scenario:}
\begin{enumerate}
\item Right after having dinner, Jeff decides to do some editing on Wikipedia. He boots up his computer and logs into Wikipedia with his account. There were some new discoveries in ornithology and he would like to adjust the information in the respective articles. A new common ancestor was found and it turns out that there's a whole other new genus now and a new actircle on that genus must be created and the genera of some birst must be adapted. 
\item Jeff prefers the VisualEditor for editing. He long switched to using Wikidata properties for his articles because it saves him time and trouble. He goes on the article of the bird he wants to edit and clicks on the VisualEditor. It boots and he clicks on the infobox which then opens the VisualEditor's pop-up. He scrolls down to the part where you fill in the genus and sees the little symbol signalising him that this information is coming form the respective Wikidata item. He can even hover over it to have this information displayed to him. He then is faced with two boxes aking him wether the information displayed has been correct at any point in time or wether it is just entirely wrong. He choses the first one because until the dicovery of the new shared ancestor it did belong to the former genus. He then is aked to type in the name of the new genus and, if known, a date when it became the new acurate term. He is also asked for a reference if there is one available. In this field he also has the option of looking at the entire list of values for that specific property if that's of relevance or interest. After pressing ok, he is informed that he now added a value to the respective property. When he now looks at the infobox the correct genus is being displayed. 
\item He then checks if there is an article for the new found common ancestor already and finds out that there isn't one, so he decides to start writing it. He first writes all the text for the article, as usual and then gets started on the infobox. When he goes to type in the name of the article it automatically suggests to use the corresponding Wikidata item instead. In this case thoudh there is no item yet so it suggests to create one. He chooses to do so and a seperate box opens up for him to add properties to the newly created item. For now these are restricted to the infoxbox templates properties. All he needs to do is fill out the property boxes as he would in the VisualEditor and press ok. They are then automatically transfered and at the same time the Item gets created so the article can refer to that Q.
\item Jeff decides that he wants to add another property which is not part of the template. He clocks on the usual button at the bottom of the VisualEditor to add another property. He types in the propertie's name and is automatically asked if he would like to add this as a value for the Wikidata item. He chooses yes and fills in the value for it. 
\item Because he used the Wikidata Item in his Infobox and text the Item and the article are now automatically likned to each other and if anyone wants to create an article about this new genus in another language he already has all the infobox information ready to use. 
\end{enumerate}
\end{document}
